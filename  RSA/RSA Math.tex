\documentclass{article}

\usepackage[]{algorithm2e}
\usepackage{mathtools}
\usepackage{amsfonts}
\usepackage{mathabx}
%opening
\title{Math Primer to Understand RSA Cryptographic Primitives}
\author{}

\begin{document}
	\maketitle
	\begin{abstract}
		In order to understand  RSA Cryptographic Primitives, we must have knowledge on modular arithmetic, finding inverse of a given number. In this article I explain basic operations in modular arithmetic, Euclidean algorithm to find  gcd of two positive integers, Extended Euclidean algorithm to find inverse of a given number.
		
	\end{abstract}
	
	\section{Modular Arithmatic}
	In modulo aruthmatic we have a modulo operator denoted by 'mod'.
	For example: 7 (mod 2) = 1, which means when 7 divided by 2 gives remainder 1.
	\subsection{Different operations in modulo arithmatic:}
	\subsubsection{Addition:}
	For example:
	   \begin{description}
	   \item[$ \bullet$] 2+1 (mod 5) = 3 (mod 5)
	   \item[$ \bullet$] 2+3 (mod 5) = 5 (mod 5) =0 (mod 5)
	   \item[$\bullet$]  2+10 (mod 5) = 12 (mod 5) = 2 (mod 5)
	    \end{description}
    \subsubsection{Subtraction:}
    For example:
     \begin{description}
    	\item[$ \bullet$] 2-1 (mod 5) = 1 (mod 5)
    	\item[$ \bullet$] 2-3 (mod 5) = -1 (mod 5) = 4 (mod 5)
    	In the above example -1 can be written as 4(i.e -1+5 = 4).
    	$ \ {\{-6, -1, 4, 9, 14 }\} $ In this group any element can be replaced by other element. Each element is generated by adding 5 to the previous element.
    \end{description}
   \subsubsection{Multiplication:}
   For example:
  \begin{description}
	\item[$ \bullet$] $4*5 (mod 5)$ = 20 (mod 5) = 0 (mod 5)
	\item[$ \bullet$] $(2+3)*7$ (mod 5) = $5*2$ (mod 5) =0 (mod 5)
     Here, for easy calculation 7 can be written as 2 (i.e 7-5=2).
\end{description}
   \subsubsection{Division:}
   $ \frac{a}{b }$(mod c) = $a*b^{-1}$(mod c).\\
   In order to calculate $b^{-1}$, it needs to satisfy the condition that gcd(b,c)=1 then $b^{-1}$ exists.

\section{Euclidean algorithm to find gcd of two positive integers}
Ex: gcd(160, 28)\\

\quad$160=5*28+20$  (Divide 160 by 28, gives remainder 20)\\

\quad$28=1*20+8$  (Divide 28 by above equation remainder 20, gives remainder value 8)\\

\quad$20=2*8+4$  (Divide 20 by above equation remainder 8, gives remainder value 4)\\

\quad$8=2*4+0$  (Divide 8 by above equation remainder 4, gives remainder value 0)\\
In the final equation when  $8\div4$ gives remainder 0.So gcd(160, 28) = 4.

\section{To find $b^{-1}$ (mod c)}
We need to find the value 'x' such that $b*x=1 $(mod c).\\
For example:
\begin{description}
	\item[$\bullet$] $6^{-1}$ (mod 7).
	
	gcd(6, 7) = 1 .
	
	so $6^{-1}$ exists. In order to find $6^{-1}$, multiply 6 with ${1,2,3,4,5,6}$
	(i.e given mod value is 7, so you could multiply 6 with integers from 1 to 6. If given mod value is 9 then you could multiply 6 with integers from 1 to 8 ).
	 
	 \quad$6*1$= 6(mod 7)
	 
	\quad$6*2$= 12(mod 7)= 5 (mod 7)
	
	\quad$6*3$= 18(mod 7)= 4 (mod 7)
	
	\quad$6*4$= 24(mod 7)= 3 (mod 7)
	
	\quad$6*5$= 30(mod 7)= 2 (mod 7)
	
	\quad$6*6$= 36(mod 7)= 1 (mod 7)
	
	\quad$6^{-1}$(mod 7)= 6(mod 7).
	\item[$\bullet$] $0^{-1}$(mod 7) does not exist.   
	
	
\end{description}

\subsection{Extended Euclidean algorithm to find inverse:}

\begin{description}
	\item 
	



Ex: Find $ 7^{-1} $(mod 19).

    Step 1:

          \quad19 = $2*7$+5 $ \rightarrow(3)$
          
         \quad7 = $1*5$+2  $\rightarrow(2)$       
         
         \quad5 = $2*2$+1   $\rightarrow(1)$      
          
          \quad2  = $2*1$+0.
          
          In the final equation when '2' is divided by '1' gives remainder '0'. So gcd(7,19)is 1 and $7^{-1}$(mod 19) exist.
          
  Step 2:
  
       Equation(1) can be rearranged as
        
      \quad$1= 5-2*2$
       
      \quad$  = 5-2(7-(1*5))\rightarrow$ from (2)
      
      
       \quad$= 5-2*7+2*5= 3*5-2*7$
      
       \quad$= 3(19-(2*7))-2*7 \rightarrow$  from (3)
       
      \quad$= 3*19-8*7   \rightarrow$(4).
      
       Take (mod 19) on both sides of equation (4), we get
       
      \quad $1(mod 19) = -7*8 (mod 19)$)
      
     \quad  $ 7^{-1} (mod 19)= -8 (mod 19)=11(mod 19)$.
   
       
       
        \end{description}
    \subsection*{Relatively prime numbers :}
    Two integers are said to be relatively prime to each other if their gcd is one. Let a,b belongs to the set of prime integers.
    If gcd(a,b)=1 then a,b are relatively prime to each other.
  
   \section{Number sets notations :}
   \begin{description}
   	\item[$\bullet$] $\mathbb{P}$ represents set of prime numbers, where $\mathbb{P}=\{ 2, 3, 5, 7... \}$.
   	\item[$\bullet$] $\mathbb{W}$ represents set of whole numbers, where $\mathbb{W}=\{ 0, 1, 2, 3... \}$.
   	\item[$\bullet$] $\mathbb{N}$ represents set of natural numbers, where $\mathbb{N}=\{ 1, 2, 3, 4 ... \}$. It is also denoted by $\mathbb{Z^{+}}$.
   	\item[$\bullet$] $\mathbb{Z}$ represents set of integers, where $\mathbb{Z}=\{...-4, -3, -2, -1, 0, 1, 2, 3, 4... \}$.
   	\item[$\bullet$] Irrational number is a real number but it can not be represented as a fraction.
   	$\mathbb{I}$ represents set of irrational numbers, Ex:$\pi$=3.14159....
   	\item[$\bullet$] Rational number is a real number that can be represented as a fraction. $\mathbb{Q}$ represents set of rational numbers.
   	 Ex:0.3333=$\frac{1}{3}$, 0.2=$\frac{1}{5}$.
   	\item[$\bullet$] Real numebrs include set of integers, set of rational  and set of irrational numbers. $\mathbb{R}$ represents set of real numbers.
   	\item[$\bullet$] Complex number is a number that can be represented in a+ib form.$\mathbb{C}$ represents set of complex numbers.
   	 where $\mathbb{C}=\{ 3+i2, i10, 1-i ... \}$.
   \end{description}
    
    \section{Euler's PHI function or Euler's totient function:}
    Let $n\in\mathbb{Z^{+}}$.
    
    The Euler's phi function,
    
    $\Phi(n)$ = number of positive integers, not greater than n, that are relatively 
    
    prime to n.
    
    
     Ex: find $\Phi(7)$
    
    gcd(1,7)=1 ;gcd(2,7)=1 ;gcd(3,7)=1 ;gcd(4,7)=1;
    gcd(5,7)=1  ;gcd(6,7)=1; gcd(7,7)=7.
    
    Therefore $\Phi(7)$=6.
    \subsubsection{Useful formulas to calculate $\Phi(n)$ :}
    \begin{description}
    	\item[$\bullet$] If n is a prime number then $\Phi(n)=n-1$.
    	\item[$\bullet$] If n is a prime number, k= 1, 2, 3...then $\Phi(n^{k})=n^{k}-n^{(k-1)}$.
    	
    %	\item[$\bullet$] If $n\in\mathbb{Z^{+}-\mathbb{1}}$
       	\item[$\bullet$] If n belongs to the set of positive integers except '1',
       	then $n=p^{\alpha_{1}}_{1}. p^{\alpha_{2}}_{2}.....p^{\alpha_{m}}_{m}$.
       	Where $p_{i}$'s are prime numbers.
       	
       $\alpha_{i}\in$ set of positive integers,  $ 1\leq i \leq m$ then
       
       $\Phi(n)= n(1-\frac{1}{p_{1}})(1-\frac{1}{p_{2}})...(1-\frac{1}{p_{m}}).$
       \item[$\bullet$]  m,n$\in$ set of positive integers, gcd(m,n)=1 then
       $\Phi(mn)=\Phi(m).\Phi(n)$.
       
      \item  Example: 1. Find $ \Phi(6)$.
       
       Sol) Let n= 6 = $3*2$.
       
      $ \Phi(6)=\Phi(3*2)=\Phi(3)*\Phi(2)=(3-1)*(2-1)=2$.
      
      \item Example: 2. Find $ \Phi(120)$.
      
      Sol) Let n= 120 = $2^{3}.3.5$.
      
      $ \Phi(120)=\Phi(2^{3}.3.5)=120*(1-\frac{1}{2})*(1-\frac{1}{3})*(1-\frac{1}{5})$
                       $=120*\frac{1}{2}*\frac{2}{3}*\frac{4}{5}=32$.   
      \item[$\bullet$] Let p and q are two co-prime numbers. If x=a (mod p)   and x=a (mod q), 
      then x= a (mod pq).
      
      Example: if 17 = 2 (mod 5), 17= 2 (mod 3) then 17 = 2 (mod 15).    	
    	
    \end{description}
\subsection{Fermat's little theorem :}
\begin{description}
	\item[$\bullet$] 'p' is a prime number, $a\in \mathbb{Z^{+}}$ and  $p \notdivides a$ (where 'a' is not divisible by 'p').
	 Then $a^{p-1} \equiv$ 1(mod p) .
	 \item[$\bullet$] 'p' is a prime number, $a\in \mathbb{W}$ then  $a^{p} \equiv$ a (mod p) .
\end{description}
\subsection{Euler's theorem or Euler - Fermat's theorem(EFT)}
EFT states that if integers a, n are relatively prime (i.e gcd(a,n)=1) then $a^{\Phi(n)}$= 1 (mod n).




    
    
    





	 
	   
	
\end{document}